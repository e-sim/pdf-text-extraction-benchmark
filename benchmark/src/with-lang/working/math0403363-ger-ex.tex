%\documentclass[12pt,leqno]{article}
\documentclass[12pt,leqno,draft]{article}


%%%%%%  a few convenient math definitions


\def\diam{\mathop{\rm diam}}
\def\dist{\mathop{\rm dist}}
\def\radius{\mathop{\rm radius}}


%%%%%%  the following commands set up theorems, lemmas, propositions, etc.,
%%%%%%  and ensure a common system of numbering for them (and for equations)

\newtheorem{theorem}{Theorem}
\newtheorem{lemma}[theorem]{Lemma}
\newtheorem{proposition}[theorem]{Proposition}
\newtheorem{definition}[theorem]{Definition}
\newtheorem{corollary}[theorem]{Corollary}


\newcommand{\begintheorem}{\addtocounter{equation}{1}\begin{theorem}}
\newcommand{\beginlemma}{\addtocounter{equation}{1}\begin{lemma}}
\newcommand{\beginproposition}{\addtocounter{equation}{1}\begin{proposition}}
\newcommand{\begindefinition}{\addtocounter{equation}{1}\begin{definition}}
\newcommand{\begincorollary}{\addtocounter{equation}{1}\begin{corollary}}

\usepackage{gb4e} %from me, ES


\begin{document}

\title{Notes on normed algebras}



\author{Stephen William Semmes	\\
	Rice University		\\
	Houston, Texas}

\date{}


\maketitle



	All vector spaces and so forth here will be defined over the
complex numbers.  If $z = x + i \, y$ is a complex number, where $x$,
$y$ are real numbers, then the complex conjugate of $z$ is denoted
$\overline{z}$ and defined to be $x - i \, y$.  The complex conjugate
of a sum or product of complex numbers is equal to the corresponding
sum or product of complex conjugates.  The modulus of a complex number
$z$ is the nonnegative real number $|z|$ such that $|z|^2$ is equal to
the product of $z$ and its complex conjugate.  Thus the modulus of a
product of complex numbers is equal to the product of their moduli,
and one can show that the modulus of a sum of two complex numbers is
less than or equal to the sum of the moduli of the complex numbers.

	By a \emph{finite-dimensional algebra} we mean a finite
dimensional complex vector space $\mathcal{A}$ equipped with a binary
operation which satisfies the usual associativity and distributivity
properties, and which has a nonzero multiplicative identity element
$e$.  In other words, $e \, x = x \, e = x$ for all $x \in
\mathcal{A}$.  Thus $\mathcal{A}$ should have positive dimension in
particular.  Notice that the multiplicative identity element $e$ is
unique.

	As a basic class of examples, let $V$ be a finite-dimensional
complex vector space with positive dimension, and consider
$\mathcal{L}(V)$, the space of linear mappings from $V$ to itself.
This is a vector space whose dimension is equal to the square of the
dimension of $V$.  It also becomes an algebra with respect to the
usual composition of linear transformations, with the identity
transformation $I$ on $V$, which sends every vector in $V$ to itself,
as the multiplicative identity element.  If $\mathcal{A}$ is any
finite-dimensional algebra, then we can identify $\mathcal{A}$ with a
subalgebra of $\mathcal{L}(\mathcal{A})$, the algebra of linear
transformations on $\mathcal{A}$ considered simply as a vector space.
Namely, each element $a$ of $\mathcal{A}$ can be identified with the
linear transformation $x \mapsto a \, x$ on $\mathcal{A}$.

	As another class of examples, let $X$ be any finite nonempty
set.  Consider the vector space of complex-valued functions on $X$.
This becomes an algebra with respect to pointwise multiplication of
functions.  The multiplicative identity element for this algebra is
the function which is equal to $1$ at each point.  Of course this
algebra is commutative.

	As a third class of examples, let $A$ be a finite semigroup
with identity element $\theta$.  Thus $A$ is a finite set, $\theta$ is
an element of $A$, and there is a binary operation on $A$ which
associates to each pair of elements $x$, $y$ of $A$ another element $x
\, y$.  This operation should be associative, so that $x (y \, z)$ and
$(x \, y) z$ should be the same for all $x$, $y$, $z$ in $A$, and it
should satisfy $\theta \, x = x \, \theta = x$ for all $x \in A$.  As
usual, $\theta$ is uniquely determined by this feature.

	Consider the vector space of complex-valued functions on $A$.
If $f_1$, $f_2$ are two such functions, then we can define their
convolution to be the function on $A$ given by
\begin{equation}
	(f_1 * f_2)(z) = \sum_{x \, y = z} f_1(x) \, f_2(y).
\end{equation}
More precisely, this sum is taken over all $x, y \in A$ such that $x
\, y = z$.  In this way the functions on $A$ becomes an algebra, using
convolution as the multiplication operation.  The multiplicative
identity element is provided by the function which is equal to $1$
at the identity element $\theta$ in $A$ and equal to $0$ at all other
elements of $A$.

	If $A$ happens to be a commutative semigroup, then the
corresponding convolution algebra will also be commutative.  For each
element $x \in A$ we can define $\delta_x$ to be the function on $A$
which is equal to $1$ at $x$ and to $0$ at other elements of $A$, and
in this way we can embedd $A$ into its own convolution algebra in such
a way the multiplication in $A$ corresponds exactly to convolution of
the corresponding functions on $A$.  These functions $\delta_x$, $x
\in A$, form a basis for the vector space of functions on $A$, and
hence the dimension of the convolution algebra of functions on $A$ is
equal to the number of elements of $A$.

	By a norm on a finite-dimensional vector space $V$ we mean a
nonnegative real-valued function $N$ on $V$ such that $N(v) = 0$ if
and only if $v = 0$, $N(v + w) \le N(v) + N(w)$ for all $v, w \in V$,
and $N(\alpha \, v) = |\alpha| \, N(v)$ for all complex numbers
$\alpha$ and $v \in V$.  If $\mathcal{A}$ is a finite-dimensional
algebra and $\|\cdot \|$ is a norm on $\mathcal{A}$ as a vector space,
then we say that $(\mathcal{A}, \|\cdot \|)$ is a normed algebra if
also $\|x \, y \| \le \|x\| \, \|y\|$ for all $x, y \in \mathcal{A}$
and the multiplicative identity element $e \in \mathcal{A}$ has norm
equal to $1$.  For instance, if $V$ is a finite-dimensional complex
vector space with finite dimension and $N$ is a norm on $V$, then the
algebra $\mathcal{L}(V)$ of linear transformations on $V$ becomes a
normed algebra with repsect to the operator norm, which associates to
a linear transformation $T$ on $V$ the maximum of $N(T(v))$ over all
$v \in V$ with $N(v) = 1$.  If $X$ is any finite nonempty set, then
the algebra of functions on $X$ becomes a normed algebra when one uses
the norm which assigns to a complex-valued function $f$ on $X$ the
maximum of $|f(x)|$ over $x \in X$.  If $A$ is a finite semigroup,
then the convolution algebra becomes a normed algebra with respect to
the norm which assigns to a function $f$ on $A$ the sum of $|f(x)|$
over $x \in A$.


%%%%%%%%%%%%%%%%%%%%%%% INSERT HERE - EJS



\ea \label{ex:culo:21}
  \ea \textit{Nach wie vor ist der Zinsüberschuß nach Risikovorsorge mit 9,7 Mrd DM die bei   weitem wichtigste Ertragskomponente. Allerdings weisen die unterschiedlichen   Steigerungsraten der einzelnen Ergebniskomponenten auf die Veränderungen im Geschäft hin. } \\
   \ex \textit{Although net interest income after provision for losses on loans and advances, at DM 9.7 billion, is still by far the most important component of income, the individual figures highlight the changes in our business.}
   \z
\z


\ea \label{ex:culo:22}
   \ea \textit{Daher setzen wir uns nachdrücklich für die Schaffung eines europäischen Systems der Finanzaufsicht ein.}  \\
    \ex \textit{Hence we expressly support the establishment of a European system of financial   supervision.} 
    \z
\z



\ea \label{ex:culo:23}
    \ea \textit{And what has happened before a few years have passed?} \\
     \ex \textit{Und was geschieht, ehe noch ein paar Jahre vergangen sind?}
     \z
\z
 

%%%%%%%%%%%%%

	Let $V$ be a finite-dimensional complex vector space of
positive dimension, and let $\mathcal{L}(V)$ denote the algebra of
linear transformations on $V$.  One can say that a linear mapping $T$
on $V$ is invertible if it is a one-to-one mapping of $V$ onto itself,
in which case the inverse of $T$ as a mapping on $V$ is also linear.
By well-known results in linear algebra $T$ is invertible if it is a
one-to-one mapping of $V$ into itself or if it maps $V$ onto itself,
because $V$ is finite-dimensional.  In algebraic terms $T$ is
invertible if there is a linear mapping $R$ on $V$ such that $R \, T$
and $T \, R$ are equal to the identity mapping on $V$.  Again because
$V$ has finite dimension, if either $R \, T$ or $T \, R$ is equal to
the identity mapping, then so is the other.

	If $T$ is a linear mapping on $V$, $I$ is the identity mapping
on $V$, and $\lambda$ is a complex number, then we get a new linear
mapping $\lambda \, I - T$.  The determinant of $\lambda \, I - T$ is
a complex number which is a polynomial in $\lambda$ of degree equal to
the dimension $n$ of $V$, and indeed the coefficient of $\lambda^n$ in
this polynomial is equal to $1$.  If $p(z)$ is any polynomial in $z$,
then we can define $p(T)$ in the usual manner, namely as the same
linear combination of powers of $T$ and the identity as we have of
powers of $z$ and $1$ in $p(z)$.  The celebrated theorem of Cayley and
Hamilton states that if $p(\lambda)$ is the polynomial given by taking
the determinant of $\lambda \, I - T$, then $p(T) = 0$.

	Of course $T$ is invertible if and only if the determinant of
$T$ is different from $0$.  If $n = 1$, then $T$ can be identified
with a complex number, and this is the same as saying that that
complex number is not equal to $0$.  When $n \ge 1$, the fact that
$p(T) = 0$ when $p(\lambda)$ is the characteristic polynomial equal to
the determinant of $\lambda I - T$ implies that if the determinant of
$T$ is different from $0$, so that the constant term of $p(\lambda)$
is different from $0$, then the inverse of $T$ can be expressed as a
polynomial of $T$ of degree $n - 1$.

	Now let $\mathcal{A}$ be a finite-dimensional algebra with
multiplicative identity element $e$.  If $x$ is an element of
$\mathcal{A}$, then we say that $x$ is invertible in $\mathcal{A}$ if
there is an element $y$ of $\mathcal{A}$ such that $y \, x = x \, y =
e$.  We can be a bit more precise and say that $x$ is left invertible
if there is an element $y_1$ of $\mathcal{A}$ such that $y_1 \, x =
e$, and that $x$ is right invertible if there is an element $y_2$ of
$\mathcal{A}$ such that $x \, y_2 = e$.  If $x$ is both left and right
invertible, with left and right inverses $R_1$, $R_2$, respectively,
then it is easy to see that $y_1 = y_2$ and $x$ is invertible.  In
particular, if $x$ is invertible, then the inverse of $x$ is unique,
and the inverse of $x$ is denoted $x^{-1}$.

	Suppose that $\mathcal{A}$ is in fact a subalgebra of
$\mathcal{L}(V)$ for some finite-dimensional vector space $V$ of
positive dimension.  As before, this can always be arranged up to
isomorphic equivalence.  If $T$ is an element of $\mathcal{A}$ which
is invertible as an element of $\mathcal{A}$, then of course $T$ is
invertible as an element of $\mathcal{L}(V)$.  Conversely, if $T$ is
invertible as an element of $\mathcal{L}(V)$, then the inverse of $T$
can be expressed as a polynomial in $T$, which therefore is an element
of $\mathcal{A}$.  In particular, it follows that $T$ is invertible if
$T$ is either left or right invertible.

	Fix a finite-dimensional algebra $\mathcal{A}$ with
multiplicative identity element $e$, and let $x$ be an element of
$\mathcal{T}$.  The spectrum of $x$ is defined to be the set of
complex numbers $\lambda$ such that $\lambda \, e - x$ does not have
an inverse in $\mathcal{A}$.  By embedding $\mathcal{A}$ into
$\mathcal{L}(V)$ for some finite-dimensional vector space $V$ we get
that the spectrum of $x$ can be described as the set of zeros of a
nonconstant polynomial on the complex numbers, and thus that the
spectrum of $x$ is a finite nonempty set of complex numbers.

	Assume that $(\mathcal{A}, \|\cdot \|)$ is a
finite-dimensional normed algebra.  Let $x$ be an element of
$\mathcal{A}$ and let $\lambda$ be a complex number such that
\begin{equation}
	|\lambda| > \|x\|.
\end{equation}
If $\mathcal{A}$ is a subalgebra of $\mathcal{L}(V)$ for some
finite-dimensional vector space $V$ and $\|\cdot \|$ is the operator
norm of a linear operator on $V$ with respect to some norm $N(\cdot )$
on $V$, then it is easy to see that $\lambda \, I - x$ has trivial
kernel as a linear operator on $V$ and hence is invertible.  In
general one can show that $\lambda \, e - x = \lambda (e -
\lambda^{-1} \, x)$ is invertible by summing the series $\sum_{j =
0}^\infty \lambda^{-j} \, x^j$.

	



\begin{thebibliography}{3}


\bibitem {1} W.~Arveson, {\it A Short Course on Spectral Theory},
Springer-Verlag, 2002.

\bibitem {2} G.~Birkhoff and S.~MacLane, {\it A Survey of Modern Algebra},
4th edition, Macmillan, 1977.

\bibitem {3} S.~MacLane and G.~Birkhoff, {\it Algebra}, 3rd edition,
AMS Chelsea Publishing, 1999.




\end{thebibliography}

\end{document}






